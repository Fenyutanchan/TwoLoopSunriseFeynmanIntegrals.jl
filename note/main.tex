% Copyright (c) 2025 Wen-Di Li <liwendi23@mails.ucas.ac.cn> and Quan-feng WU <wuquanfeng@ihep.ac.cn>
% 
% This software is released under the MIT License.
% https://opensource.org/licenses/MIT

\documentclass{article}

\usepackage{amsmath}
% \usepackage{amssymb}
\usepackage{authblk}
\usepackage[sorting=none, style=numeric-comp]{biblatex}
    \addbibresource{from_inspirehep.bib}
    \addbibresource{others.bib}
\usepackage{fontawesome5}
\usepackage[a4paper, margin=2cm]{geometry}
\usepackage{graphicx}
\usepackage{xcolor}
\usepackage[
    colorlinks=true,
    anchorcolor=violet,
    citecolor=orange,
    linkcolor=blue,
    urlcolor=magenta
]{hyperref}
% \usepackage[color=violet]{attachfile2}
% \usepackage{mathrsfs}
\usepackage{mathtools}
\usepackage{pgfplots}
    \pgfplotsset{compat=1.18}
\usepackage{physics}
\usepackage{slashed}
\usepackage{subcaption}
\usepackage{tensor}
\usepackage[compat=1.1.0]{tikz-feynhand}
    % \tikzset{graviton/.style={decorate, decoration={snake, amplitude=.4mm, segment length=1.5mm, pre length=.5mm, post length=.5mm}, double}}

\newcommand{\email}[1]{\footnote{E-mail: \href{mailto:#1}{#1}}}
\newcommand{\eg}{\textit{e.g.}}
\newcommand{\etc}{\textit{etc.}}
\newcommand{\ie}{\textit{i.e.}}
\newcommand{\license}{\footnote{This note © 2025 by \href{https://github.com/Fenyutanchan}{Quan-feng WU} is licensed under \href{https://creativecommons.org/licenses/by-nc-/4.0/}{CC BY-NC-ND 4.0}. \faCreativeCommons \faCreativeCommonsBy \faCreativeCommonsNc \faCreativeCommonsNd}}

\DeclareDocumentCommand{\TBA}{o}{\IfNoValueTF{#1}{{\color{red} TBA.}}{{\color{red} TBA: #1}}}
\DeclareDocumentCommand{\TBD}{o}{\IfNoValueTF{#1}{{\color{red} TBD.}}{{\color{red} TBD: #1}}}

\DeclareDocumentCommand{\githubsrc}{o}{\IfNoValueTF{#1}{\url{https://github.com/Fenyutanchan/TwoLoopSunriseFeynmanIntegrals.jl.git}}{\href{https://github.com/Fenyutanchan/TwoLoopSunriseFeynmanIntegrals.jl/blob/main/#1}{\texttt{#1}}}}

\newcommand{\notsurecolor}{green!50!black!100!}
\newcommand{\notsure}[1]{{\color{\notsurecolor{}} #1}}

\newcommand{\ee}{\mathrm{e}}
\newcommand{\ii}{\mathrm{i}}

\newcommand{\eulergamma}{\gamma_\mathrm{E}}

\DeclareDocumentCommand{\twopi}{o}{\IfNoValueTF{#1}{2 \pi}{\qty(2 \pi)^{#1}}}
\DeclareDocumentCommand{\diracdelta}{o}{\IfNoValueTF{#1}{\delta}{\delta^{(#1)}}}
\DeclareDocumentCommand{\twopidelta}{om}{\IfNoValueTF{#1}{\twopi[] \diracdelta}{\twopi[#1] \diracdelta[#1]} \qty(#2)}
\DeclareDocumentCommand{\TSI}{mo}{I_\mathrm{TSI}^{\qty(#1)}\IfNoValueF{#2}{\qty(#2)}}
\DeclareDocumentCommand{\Ione}{mo}{I_1^{\qty(#1)}\IfNoValueF{#2}{\qty(#2)}}
\DeclareDocumentCommand{\lambdaFun}{om}{\lambda\IfNoValueF{#1}{^{#1}}\qty(#2)}

\DeclareDocumentCommand{\Li}{oom}{\mathrm{Li}\IfNoValueF{#1}{_{#1}}\IfNoValueF{#2}{^{#2}}\qty(#3)}

\DeclareDocumentCommand{\denominator}{moo}{\qty(-#1^2 \IfNoValueF{#2}{+ #2^2}) \IfNoValueF{#3}{^{#3}}}
\DeclareDocumentCommand{\qmOne}{so}{\denominator{q_1}[m_1][\IfBooleanTF{#1}{\nu_1}{\IfNoValueF{#2}{#2}}]}
\DeclareDocumentCommand{\qmTwo}{so}{\denominator{q_2}[m_2][\IfBooleanTF{#1}{\nu_2}{\IfNoValueF{#2}{#2}}]}
\DeclareDocumentCommand{\qmThree}{so}{\denominator{q_{12}}[m_3][\IfBooleanTF{#1}{\nu_3}{\IfNoValueF{#2}{#2}}]}

\newcommand{\tildedp}[1]{\widetilde{\dd{#1}}}

% copied from https://tex.stackexchange.com/questions/2476/using-latex-to-render-hypergeometric-function-notation
\newmuskip\pFqmuskip
\newcommand*\pFq[6][8]{%
    \begingroup % only local assignments
        \pFqmuskip=#1mu\relax
        \mathchardef\normalcomma=\mathcode`,
        % make the comma math active
        \mathcode`\,=\string"8000
        % and define it to be \pFqcomma
        \begingroup\lccode`\~=`\,
        \lowercase{\endgroup\let~}\pFqcomma
        % typeset the formula
        {}_{#2}F_{#3}{\left(\genfrac..{0pt}{}{#4}{#5}\middle|#6\right)}%
    \endgroup
}
\newcommand{\pFqcomma}{{\normalcomma}\mskip\pFqmuskip}




\title{Note of \texttt{TwoLoopSunriseFeynmanIntegrals.jl}}

\author{Wen-Di Li\email{liwendi23@mails.ucas.ac.cn}}
\affil{
    School of Fundamental Physics and Mathematical Sciences, \\
    Hangzhou Institute for Advanced Study, \\
    University of Chinese Academy of Sciences, \\
    Hangzhou 310024, CHINA
}

\author{Quan-feng WU\email{wuquanfeng@ihep.ac.cn}}
\affil{
    Institute of High Energy Physics, \\
    Chinese Academy of Sciences, \\
    Beijing 100049, CHINA
}

\date{\today\license}

\begin{document}
    \maketitle

    \begin{abstract}
        \TBA{}
    \end{abstract}
    \noindent\hrulefill

    \tableofcontents
    \clearpage

    \section{Introduction}

        We provide a \textsc{Julia}\footnote{\url{https://julialang.org}} package \texttt{TwoLoopSunriseFeynmanIntegrals.jl}\footnote{\githubsrc} for two-loop sunrise Feynman integrals\footnote{We suggest Refs.~\cite{Smirnov:2012gma, Weinzierl:2022eaz} for pedagogical introduction to Feynman integrals.}, which reads \cite[Eq.~(2.56)]{Weinzierl:2022eaz}\footnote{For simplicity, the prefactor $\ee^{2 \varepsilon \eulergamma} \qty(\mu^2)^{\nu - d}$ with $\nu \equiv \nu_1 + \nu_2 + \nu_3$ for the \emph{modified minimal subtraction scheme} is omitted here.}
        \begin{equation}
            \TSI{\nu_1, \nu_2, \nu_3}[d, m_1, m_2, m_3] \coloneq \int \frac{\tildedp{q_1} ~ \tildedp{q_2}}{\qmOne* \qmTwo* \qmThree*},
        \end{equation}
        where $d = 4 - 2 \varepsilon$ is the spacetime dimension, $m_i$'s are the masses, $\nu_i$'s are the exponents, and $q_i$ are the loop momenta ($q_{12} \equiv q_1 + q_2$).
        In this note, we take the loop momentum measure as
        \begin{equation}
            \tildedp{q} \coloneq \frac{\dd[d]{q}}{\ii \pi^{d/2}}.
        \end{equation}
        One can easily verify that that this integral is invariant under the permutation of $\qty(m_1, \nu_1)$, $\qty(m_2, \nu_2)$, and $\qty(m_3, \nu_3)$, \eg,
        \begin{equation}
            \TSI{\nu_1 \nu_2 \nu_3}[d, m_1, m_2, m_3] \equiv \TSI{\nu_2 \nu_3 \nu_1}[d, m_2, m_3, m_1].
        \end{equation}
        If all masses vanish, the integral vanishes as
        \begin{equation}
            \TSI{\nu_1 \nu_2 \nu_3}[d, 0, 0, 0] \equiv 0
            \label{eq:all-masses-vanishing}
        \end{equation}
        since the definition of the Feynman integral in the dimensional regularization \cite[Sec.~2.4.2]{Weinzierl:2022eaz}.

        This note is organized as follows:
        In Sec.~\ref{sec:reduction}, we reduce the two-loop sunrise Feynman integrals to the master integrals (MIs) via integration-by-part (IBP) techniques.
        In Sec.~\ref{sec:master-integral}, the MIs are evaluated and expanded into $\varepsilon$-series.
        % In Sec.~\ref{sec:implementation}, we introduce the implementation of the package.
        Finally, we conclude in Sec.~\ref{sec:conclusion}.
    
    \section{Integration-by-Part Reduction}\label{sec:reduction}

        In this section, we consider the integration-by-part (IBP) reduction for the two-loop sunrise Feynman integrals.
        The IBP reduction starts from the fact that \cite[Eq.~(6.2)]{Weinzierl:2022eaz}
        \begin{equation}
            \int \tildedp{q} ~ \pdv{q^\mu} \qty[k^\mu \cdots] \equiv 0,
        \end{equation}
        where $k$ is an arbitrary momentum.
        We also have
        \begin{align}
            \pdv{q^\mu} \frac{1}{\denominator{q}[m][\nu]} & = \frac{2 \nu q_\mu}{\denominator{q}[m][\nu + 1]}, \\
            \pdv{q_1^\mu} \frac{1}{\qmThree*} & = \frac{2 \nu_3 \qty(q_{12})_\mu}{\qmThree[\nu_3 + 1]}.
        \end{align}
        Therefore,
        \begin{equation}
            \begin{aligned}
                0 & = & & \int \tildedp{q_1} ~ \tildedp{q_2} ~ \pdv{q_1^\mu} \frac{q_1^\mu}{\qmOne* \qmTwo* \qmThree*} \\
                & = & & \qty(d - 2 \nu_1 - \nu_3) \TSI{\nu_1 \nu_2 \nu_3} + 2 \nu_1 m_1^2 \TSI{\nu_1 + 1, \nu_2, \nu_3} - \nu_3 \qty(\TSI{\nu_1 - 1, \nu_2, \nu_3 + 1} - \TSI{\nu_1, \nu_2 - 1, \nu_3 + 1}) \\
                & & & {} + \nu_3 \qty(m_1^2 - m_2^2 + m_3^2) \TSI{\nu_1, \nu_2, \nu_3 + 1}.
            \end{aligned}
        \end{equation}
        Or equivalently,
        \begin{equation}
            \begin{aligned}
                -2 \nu_1 m_1^2 \TSI{\nu_1 + 1, \nu_2, \nu_3} - \nu_3 \qty(m_1^2 - m_2^2 + m_3^2) \TSI{\nu_1, \nu_2, \nu_3 + 1} & = & & \qty(d - 2 \nu_1 - \nu_3) \TSI{\nu_1 \nu_2 \nu_3} \\
                & & & {} - \nu_3 \qty(\TSI{\nu_1 - 1, \nu_2, \nu_3 + 1} - \TSI{\nu_1, \nu_2 - 1, \nu_3 + 1}). \\
            \end{aligned}
            \label{eq:IBP-1}
        \end{equation}
        Similarly, we have
        \begin{equation}
            \begin{aligned}
                -2 \nu_2 m_2^2 \TSI{\nu_1, \nu_2 + 1, \nu_3} - \nu_3 \qty(m_2^2 - m_1^2 + m_3^2) \TSI{\nu_1, \nu_2, \nu_3 + 1} & = & & \qty(d - 2 \nu_2 - \nu_3) \TSI{\nu_1 \nu_2 \nu_3} \\
                & & & {} - \nu_3 \qty(\TSI{\nu_1, \nu_2 - 1, \nu_3 + 1} - \TSI{\nu_1 - 1, \nu_2, \nu_3 + 1}). \\
            \end{aligned}
            \label{eq:IBP-2}
        \end{equation}
        Now consider
        \begin{equation}
            \begin{aligned}
                0 & = & & \int \tildedp{q_1} ~ \tildedp{q_2} ~ \pdv{q_2^\mu} \frac{q_1^\mu}{\qmOne* \qmTwo* \qmThree*} \\
                & = & & \qty(\nu_2 - \nu_3) \TSI{\nu_1 \nu_2 \nu_3} + \nu_2 \TSI{\nu_1 - 1, \nu_2 + 1, \nu_3} - \nu_2 \TSI{\nu_1, \nu_2 + 1, \nu_3 - 1} + \nu_2 \qty(-m_1^2 - m_2^2 + m_3^2) \TSI{\nu_1, \nu_2 + 1, \nu_3} \\
                & & & {} + \nu_3 \TSI{\nu_1, \nu_2 - 1, \nu_3 + 1} - \nu_3 \TSI{\nu_1 - 1, \nu_2, \nu_3 + 1} + \nu_3 \qty(m_1^2 - m_2^2 + m_3^2) \TSI{\nu_1, \nu_2, \nu_3 + 1},
            \end{aligned}
        \end{equation}
        or equivalently,
        \begin{equation}
            \begin{aligned}
                & & & \nu_2 \qty(m_1^2 + m_2^2 - m_3^2) \TSI{\nu_1, \nu_2 + 1, \nu_3} - \nu_3 \qty(m_1^2 - m_2^2 + m_3^2) \TSI{\nu_1, \nu_2, \nu_3 + 1} \\
                & = & & \qty(\nu_2 - \nu_3) \TSI{\nu_1 \nu_2 \nu_3} + \nu_2 \qty(\TSI{\nu_1 - 1, \nu_2 + 1, \nu_3} - \TSI{\nu_1, \nu_2 + 1, \nu_3 - 1}) + \nu_3 \qty(\TSI{\nu_1, \nu_2 - 1, \nu_3 + 1} - \TSI{\nu_1 - 1, \nu_2, \nu_3 + 1}).
            \end{aligned}
            \label{eq:IBP-3}
        \end{equation}
        Combining Eqs.~\eqref{eq:IBP-1}, \eqref{eq:IBP-2}, and \eqref{eq:IBP-3}, we have
        \begin{equation}
            \begin{aligned}
                & & & \mqty*(
                    -2 \nu_1 m_1^2 & 0 & -\nu_3 \qty(m_1^2 - m_2^2 + m_3^2) \\
                    0 & -2 \nu_2 m_2^2 & -\nu_3 \qty(m_2^2 - m_1^2 + m_3^2) \\
                    0 & \nu_2 \qty(m_1^2 + m_2^2 - m_3^2) & -\nu_3 \qty(m_1^2 - m_2^2 + m_3^2)
                ) \mqty*(\TSI{\nu_1 + 1, \nu_2, \nu_3} \\ \TSI{\nu_1, \nu_2 + 1, \nu_3} \\ \TSI{\nu_1, \nu_2, \nu_3 + 1}) \\
                & = & & \mqty*(
                    d - 2 \nu_1 - \nu_3 & 0 & \nu_3 & 0 & -\nu_3 \\
                    d - 2 \nu_2 - \nu_3 & 0 & -\nu_3 & 0 & \nu_3 \\
                    \nu_2 - \nu_3 & -\nu_2 & \nu_3 & \nu_2 & -\nu_3
                ) \mqty*(
                    \TSI{\nu_1 \nu_2 \nu_3} \\
                    \TSI{\nu_1, \nu_2 + 1, \nu_3 - 1} \\ \TSI{\nu_1, \nu_2 - 1, \nu_3 + 1} \\
                    \TSI{\nu_1 - 1, \nu_2 + 1, \nu_3} \\ \TSI{\nu_1 - 1, \nu_2, \nu_3 + 1}
                ).
            \end{aligned}
        \end{equation}
        Defining the matrices $\vb{A}$ and $\vb{B}$ as
        \begin{align}
            \vb{A} & \coloneq \mqty*(
                -2 \nu_1 m_1^2 & 0 & -\nu_3 \qty(m_1^2 - m_2^2 + m_3^2) \\
                0 & -2 \nu_2 m_2^2 & -\nu_3 \qty(m_2^2 - m_1^2 + m_3^2) \\
                0 & \nu_2 \qty(m_1^2 + m_2^2 - m_3^2) & -\nu_3 \qty(m_1^2 - m_2^2 + m_3^2)
            ), \\
            \vb{B} & \coloneq \mqty*(
                d - 2 \nu_1 - \nu_3 & 0 & \nu_3 & 0 & -\nu_3 \\
                d - 2 \nu_2 - \nu_3 & 0 & -\nu_3 & 0 & \nu_3 \\
                \nu_2 - \nu_3 & -\nu_2 & \nu_3 & \nu_2 & -\nu_3
            ),
        \end{align}
        the solution is given by
        \begin{equation}
            \mqty*(\TSI{\nu_1 + 1, \nu_2, \nu_3} \\ \TSI{\nu_1, \nu_2 + 1, \nu_3} \\ \TSI{\nu_1, \nu_2, \nu_3 + 1}) = \vb{A}^{-1} \vb{B} \mqty*(
                \TSI{\nu_1 \nu_2 \nu_3} \\
                \TSI{\nu_1, \nu_2 + 1, \nu_3 - 1} \\ \TSI{\nu_1, \nu_2 - 1, \nu_3 + 1} \\
                \TSI{\nu_1 - 1, \nu_2 + 1, \nu_3} \\ \TSI{\nu_1 - 1, \nu_2, \nu_3 + 1}
            )
            \label{eq:IBP-solution}
        \end{equation}
        if $\det \vb{A} \neq 0$.
        Notice that the determinant of $\vb{A}$ is given by
        \begin{equation}
            \det \vb{A} = 2 \nu_1 \nu_2 \nu_3 m_1^2 \lambdaFun{m_1^2, m_2^2, m_3^2},
            \label{eq:det-A}
        \end{equation}
        where $\lambdaFun{x, y, z}$ is the Källén triangle function \cite[Eq.~(6.3)--(6.7)]{Byckling:1971ParticleKinematics}
        \begin{equation}
            \begin{aligned}
                \lambdaFun{x, y, z} & \coloneq x^2 + y^2 + z^2 - 2 x y - 2 y z - 2 z x \\
                & = \qty(\sqrt{x} + \sqrt{y} + \sqrt{z}) \qty(\sqrt{x} + \sqrt{y} - \sqrt{z}) \qty(\sqrt{x} - \sqrt{y} + \sqrt{z}) \qty(-\sqrt{x} + \sqrt{y} + \sqrt{z}).
            \end{aligned}
            \label{eq:Kallen-triangle}
        \end{equation}
        Hence, there are two cases --- non-collinear and collinear --- to be considered.

        \subsection{Non-Collinear Case}

            In the non-collinear case, the solution in Eq.~\eqref{eq:IBP-solution} is valid, which can be used to eliminate the sum of $\nu_i$'s by one from the left-hand-side (LHS) to the right-hand-side (RHS) of Eq.~\eqref{eq:IBP-solution}, which is the key to reduce the two-loop sunrise Feynman integrals to the master integrals (MIs).
            The expression of $\vb{A}^{-1} \vb{B}$ is too lengthy to be presented here, but it could be reproduced by the \textsc{Wolfram Mathematica}\footnote{\url{https://www.wolfram.com/mathematica}} notebook \githubsrc[note/Mathematica\_notebooks/IBP\_NC.nb].
            The generated expressions are stored in the directory \githubsrc[ext/ibp\_nc/] for further use.

            As Eq.~\eqref{eq:det-A} shows,
            \begin{equation}
                \begin{aligned}
                    & & & \det \vb{A} = 0 \\
                    & \Leftrightarrow & & \nu_1 = 0 \lor \nu_2 = 0 \lor \nu_3 = 0 \lor \lambdaFun{m_1^2, m_2^2, m_3^2} = 0.
                \end{aligned}
            \end{equation}
            with $m_1 > 0$\footnote{
                If $m_1 = 0$, we can just apply the permutation of $\qty(m_1, \nu_1)$, $\qty(m_2, \nu_2)$, and $\qty(m_3, \nu_3)$ to satisfy the condition except for the case of $m_1 = m_2 = m_3 = 0$.
                However the case of $m_1 = m_2 = m_3 = 0$ is evaluated to $0$ as Eq.~\eqref{eq:all-masses-vanishing} shows.
            }, $m_2 \ge 0$, and $m_3 \ge 0$.
            The part of $\lambdaFun{m_1^2, m_2^2, m_3^2} = 0$ is so-called collinear condition since it can be factorized as [Eq.~\eqref{eq:Kallen-triangle}]
            \begin{equation}
                \lambdaFun{m_1^2, m_2^2, m_3^2} = \qty(m_1 + m_2 + m_3) \qty(m_1 + m_2 - m_3) \qty(m_1 - m_2 + m_3) \qty(-m_1 + m_2 + m_3).
            \end{equation}
            For the part of $\nu_1 = 0 \lor \nu_2 = 0 \lor \nu_3 = 0$, there are several cases to be considered.
            
            \paragraph{\boldmath $\nu_1 = \nu_2 = \nu_3 = 0$.}
            The integrals are trivial as
            \begin{equation}
                \TSI{0 0 0}[d, m_1, m_2, m_3] = \int \tildedp{q_1} ~ \tildedp{q_2} \equiv 0.
                \label{eq:TSI-000}
            \end{equation}

            \paragraph{\boldmath $\nu_2 = \nu_3 = 0$.}
            The integrals are also trivial as
            \begin{equation}
                \TSI{0 0 \nu_3} = \int \frac{\tildedp{q_1} ~ \tildedp{q_2}}{\qmOne*} = \int \tildedp{q_2} \int \frac{\tildedp{q_1}}{\qmOne*} \equiv 0.
                \label{eq:TSI-00n}
            \end{equation}
            For the cases of $\nu_3 = \nu_1 = 0$ or $\nu_1 = \nu_2 = 0$, the permutation of $\qty(m_1, \nu_1)$, $\qty(m_2, \nu_2)$, and $\qty(m_3, \nu_3)$ can be applied to obtain the same results.

            \paragraph{\boldmath $\nu_3 = 0$.}
            The integrals are factorized as
            \begin{equation}
                \begin{aligned}
                    \TSI{\nu_1 \nu_2 0} = \int \frac{\tildedp{q_1} ~ \tildedp{q_2}}{\qmOne* \qmTwo*} & = \int \frac{\tildedp{q_1}}{\qmOne*} \int \frac{\tildedp{q_2}}{\qmTwo*} \\
                    & \equiv \Ione{\nu_1}[d, m_1] \Ione{\nu_2}[d, m_2],
                \end{aligned}
                \label{eq:TSI-factorization}
            \end{equation}
            where $\Ione{\nu}[d, m]$ is evaluateed to \cite[Eq.~(10.1)]{Smirnov:2012gma}
            \begin{equation}
                \Ione{\nu}[d, m] = \frac{\Gamma\qty(\nu - \frac{d}{2})}{\Gamma(\nu)} \qty(m^2)^{\frac{d}{2} - \nu}.
                \label{eq:I1}
            \end{equation}
            For the cases of $\nu_1 = 0$ or $\nu_2 = 0$, the permutation of $\qty(m_1, \nu_1)$, $\qty(m_2, \nu_2)$, and $\qty(m_3, \nu_3)$ can be applied to obtain the same results.

            The integrals with $\nu_1 = 0 \lor \nu_2 = 0 \lor \nu_3 = 0$ are called the boundary cases.
            For every integral with $\nu_1 \ge 0 \land \nu_2 \ge 0 \land \nu_3 \ge 0$, the IBP reduction can be applied to reduce the integrals to the boundary cases and $\TSI{111}$.
            Actually, there are only two MIs in the non-collinear case --- $\TSI{111}$ and $\TSI{110}$.
            However, the boundary cases are easy to evaluate as Eq.~\eqref{eq:I1} shows, so we do not reduce $\TSI{\nu_1 \nu_2 0}$ to $\TSI{110}$ in practice.

        \subsection{Collinear Case}

            If $\lambdaFun{m_1^2, m_2^2, m_3^2} = 0$, we cannot apply Eq.~\eqref{eq:IBP-solution} to reduce the integrals since $\vb{A}$ is singular, \ie, $\vb{A}^{-1}$ does not exist.
            Notice that
            \begin{equation}
                \lambda(m_1^2, m_2^2, m_3^2) = 0 \quad \Longleftrightarrow \quad m_1 + m_2 + m_3 = 0 \lor m_1 = m_2 + m_3 \lor m_2 = m_1 + m_3 \lor m_3 = m_1 + m_2.
            \end{equation}
            Since $m_i \ge 0$ and $\TSI{\nu_1 \nu_2 \nu_3}[d, 0, 0, 0]$ vanishes as Eq.~\eqref{eq:all-masses-vanishing} shows, the first condition is excluded.
            The other three conditions are called the collinear cases.
            Without loss of generality, the permutation of $\qty(m_1, \nu_1)$, $\qty(m_2, \nu_2)$, and $\qty(m_3, \nu_3)$ can be applied to obtain the case $m_1 = m_2 + m_3$.

            From Eq.~(2.4) of Ref.~\cite{Davydychev:2022dcw}, we have
            \begin{equation}
                \TSI{\nu_1, \nu_2, \nu_3}[d, m_1, m_2, m_3] = \frac{1}{2 \qty(d + 3 - 2 \nu) m_1 m_2 m_3} \qty{
                    \begin{aligned}
                        \qty[m_1 \qty(d + 2 - \nu) - m_2 \nu_3 - m_3 \nu_2] \TSI{\nu_1 - 1, \nu_2, \nu_3} \\
                        {} + \qty[m_1 \nu_3 - m_2 \qty(d + 2 - \nu) - m_3 \nu_1] \TSI{\nu_1, \nu_2 - 1, \nu_3} \\
                        {} + \qty[m_1 \nu_2 - m_2 \nu_1 - m_3 \qty(d + 2 - \nu)] \TSI{\nu_1, \nu_2, \nu_3 - 1} \\
                    \end{aligned}
                },
            \end{equation}
            where $m_1 = m_2 + m_3$ and $\nu = \nu_1 + \nu_2 + \nu_3$.
            We can apply this formula to reduce the collinear cases to the boundary cases, \ie, $\nu_i = 0$ for any $i = 1, 2, 3$.
            The boundary cases of any $\nu_i = 0$ are the same as Eqs.~\eqref{eq:TSI-000}, \eqref{eq:TSI-00n}, and \eqref{eq:TSI-factorization}.
            We also perform the $\varepsilon$-expansion in \githubsrc[note/Mathematica\_notebooks/IBP\_CL.nb], and the results are stored in the directory \githubsrc[ext/ibp\_cl/] for further use.
            
            Different from the non-collinear case, $\TSI{111}$ is no longer the boundary case (or MIs) that needs to be evaluated independently.
            Instead, we have \cite[Eq.~(2.14)]{Davydychev:2022dcw}
            \begin{equation}
                \TSI{111}[d, m_1, m_2, m_3] = \frac{d - 2}{2 (d - 3)} \qty(\frac{\TSI{011}}{m_2 m_3} - \frac{\TSI{101}}{m_3 m_1} - \frac{\TSI{110}}{m_1 m_2})
            \end{equation}
            for $m_1 = m_2 + m_3$.

    \section{Master Integrals}\label{sec:master-integral}
            
        In this section, we evaluate the MIs and expand them into $\varepsilon$-series.

        \subsection{Factorization as Products of One-Loop Integrals}

            As Eq.~\eqref{eq:TSI-factorization} shows, the factorization of $\TSI{\nu_1 \nu_2 0} = \Ione{\nu_1} \Ione{\nu_2}$ makes the evaluation of $\TSI{\nu_1 \nu_2 0}$ easier.
            The expansion of $\Ione{\nu}[d, m]$ into $\varepsilon$-series is easily obtained as
            \begin{align}
                \Ione{1}[d, m] & = -\frac{m^2}{\varepsilon} - m^2 \qty(1 - \eulergamma - \ln m^2) + \order{\varepsilon^1}, \\
                \Ione{2}[d, m] & = \frac{1}{\varepsilon} - \qty(\eulergamma + \ln m^2) + \order{\varepsilon^1}, \\
                \Ione{\nu > 2}[d, m] & = \frac{\qty(m^2)^{2 - \nu}}{\qty(\nu - 1) \qty(\nu - 2)} + \order{\varepsilon^1}.
            \end{align}
            These expansions are evaluated in \githubsrc[note/Mathematica\_notebooks/I\_1-loop.nb] and stored in the directory \githubsrc[ext/one\_loop/] for further use.

        \subsection{\boldmath Non-Collinear \texorpdfstring{$\TSI{111}$}{TSI-111}}

            \paragraph{All non-vanishing masses.}
            For the master integral $\TSI{111}$, Eq.~(4.9) of Ref.~\cite{Davydychev:1992mt} shows
            \begin{equation}
                \begin{aligned}
                    \TSI{111}[d, m_1, m_2, m_3] & = \frac{1}{2} \qty(m_1^2)^{1 - 2 \varepsilon} A(\varepsilon) \qty[
                        \begin{aligned}
                            -\frac{1}{\varepsilon^2} \qty(1 + x + y) + \frac{2}{\varepsilon} \qty(x \ln x + y \ln y) - x \ln^2 x - y \ln^2 y \\
                            {} + (1 - x - y) \ln x \ln y - \lambdaFun{1, x, y} \Phi(x, y)
                        \end{aligned}
                    ]
                \end{aligned}
                \label{eq:TSI-111}
            \end{equation}
            where $m_1, m_2, m_3 > 0$,
            \begin{equation}
                \left\{
                    \begin{aligned}
                        x & \coloneq \frac{m_2^2}{m_1^2}, \\
                        y & \coloneq \frac{m_3^2}{m_1^2},
                    \end{aligned}
                \right.
            \end{equation}
            with $0 \le x, y \le 1$, and
            \begin{equation}
                A(\varepsilon) \coloneq \frac{\Gamma^2\qty(1 + \varepsilon)}{(1 - \varepsilon) (1 - 2 \varepsilon)}.
            \end{equation}
            
            \begin{enumerate}
                \item For $\lambdaFun{1, x, y} > 0$, we have $\sqrt{x} + \sqrt{y} < 1$ and
                    \begin{equation}
                        \Phi(x, y) \coloneq \frac{1}{\sqrt{\lambdaFun{1, x, y}}} \qty[
                            \begin{aligned}
                                2 \ln\qty(\frac{1 + x - y - \sqrt{\lambdaFun{1, x, y}}}{2}) \ln\qty(\frac{1 - x + y - \sqrt{\lambdaFun{1, x, y}}}{2}) \\
                                {} - 2 \Li[2]{\frac{1 - x + y - \sqrt{\lambdaFun{1, x, y}}}{2}} - 2 \Li[2]{\frac{1 + x - y - \sqrt{\lambdaFun{1, x, y}}}{2}} \\
                                {} - \ln x \ln y + \frac{\pi^2}{3}
                            \end{aligned}
                        ];
                    \end{equation}
                \item For $\lambdaFun{1, x, y} < 0$, we have $\sqrt{x} + \sqrt{y} > 1$ and
                    \begin{equation}
                        \Phi(x, y) \coloneq \frac{2}{\sqrt{-\lambdaFun{1, x, y}}} \qty[
                            \begin{aligned}
                                \operatorname{Cl}_2\qty(2 \arccos \frac{1 + x - y}{2 \sqrt{x}}) + \operatorname{Cl}_2\qty(2 \arccos \frac{1 - x + y}{2 \sqrt{y}}) \\
                                {} + \operatorname{Cl}_2\qty(2 \arccos \frac{-1 + x + y}{2 \sqrt{x y}}).
                            \end{aligned}
                        ]
                    \end{equation}
            \end{enumerate}
            The $\varepsilon$-series expansions of $\TSI{111}$ are evaluated in \githubsrc[note/Mathematica\_notebooks/TSI111\_NC.nb] and stored in the directory \githubsrc[ext/TSI111\_nc/] for further use.

            \paragraph{One vanishing mass.}
            For the case of $m_1 > m_2 > m_3 = 0$, Eq.~(31) in Ref.~\cite{Fleischer:1994ef} gives that
            \begin{equation}
                \begin{aligned}
                    \TSI{111}[d, m_1, m_2, 0] & = \frac{\Gamma\qty(3 - \frac{d}{2}) \Gamma\qty(\frac{d}{2} - 1) \Gamma^2\qty(2 - \frac{d}{2})}{\Gamma^2(1) \Gamma\qty(\frac{d}{2}) \Gamma\qty(4 - d) \qty(m_1^2)^{3 - d}} \pFq{2}{1}{3 - d, 2 - \frac{d}{2}}{4 - d}{1 - \frac{m_2^2}{m_1^2}} \\
                    & = \frac{m_1^2 + m_2^2}{\varepsilon} + \qty(1 - \eulergamma) \qty(m_1^2 + m_2^2) - 2 \qty(m_1^2 \ln m_1^2 + m_2^2 \ln m_2^2) + \order{\varepsilon},
                \end{aligned}
            \end{equation}
            which is evaluated in \githubsrc[note/Mathematica\_notebooks/TSI111\_NC\_one\_vanishing.nb] and stored in the directory \githubsrc[ext/TSI111\_nc/] for further use.

            \paragraph{Two vanishing masses.}
            For the case of $m_1 > m_2 = m_3 = 0$, Eq.~(10.39) of Ref.~\cite{Smirnov:2012gma} gives that
            \begin{equation}
                \begin{aligned}
                    \TSI{111}[d, m_1, 0, 0] & = & & \frac{\Gamma\qty(3 - d)}{\qty(m_1^2)^{3 - d}} \frac{\Gamma\qty(2 - \frac{d}{2}) \Gamma^2\qty(\frac{d}{2} - 1)}{\Gamma^3(1) \Gamma\qty(\frac{d}{2})} \\
                    & = & & -\frac{m_1^2}{2 \varepsilon^2} + \frac{m_1^2}{2 \varepsilon} \qty(-3 + 2 \eulergamma + 2 \ln m_1^2) \\
                    & & & {} + m_1^2 \qty[-\frac{7}{2} + \qty(3 - \eulergamma) \eulergamma - \frac{\pi^2}{4} + \qty(3 - 2 \eulergamma) \ln m_1^2 - \ln^2 m_1^2] + \order{\varepsilon},
                \end{aligned}
            \end{equation}
            which is evaluated in \githubsrc[note/Mathematica\_notebooks/TSI111\_NC\_two\_vanishing.nb] and stored in the directory \githubsrc[ext/TSI111\_nc/] for further use.

    % \section{Implementation}\label{sec:implementation}

    \section{Conclusion}\label{sec:conclusion}

    \clearpage
    \printbibliography[heading=bibintoc]
\end{document}

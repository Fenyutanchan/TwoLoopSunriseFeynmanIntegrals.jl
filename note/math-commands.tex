% Copyright (c) 2025 Wen-Di Li <liwendi23@mails.ucas.ac.cn> and Quan-feng WU <wuquanfeng@ihep.ac.cn>
% 
% This software is released under the MIT License.
% https://opensource.org/licenses/MIT

\newcommand{\ee}{\mathrm{e}}
\newcommand{\ii}{\mathrm{i}}

\newcommand{\eulergamma}{\gamma_\mathrm{E}}

\DeclareDocumentCommand{\twopi}{o}{\IfNoValueTF{#1}{2 \pi}{\qty(2 \pi)^{#1}}}
\DeclareDocumentCommand{\diracdelta}{o}{\IfNoValueTF{#1}{\delta}{\delta^{(#1)}}}
\DeclareDocumentCommand{\twopidelta}{om}{\IfNoValueTF{#1}{\twopi[] \diracdelta}{\twopi[#1] \diracdelta[#1]} \qty(#2)}
\DeclareDocumentCommand{\TSI}{mo}{I_\mathrm{TSI}^{\qty(#1)}\IfNoValueF{#2}{\qty(#2)}}
\DeclareDocumentCommand{\Ione}{mo}{I_1^{\qty(#1)}\IfNoValueF{#2}{\qty(#2)}}
\DeclareDocumentCommand{\lambdaFun}{om}{\lambda\IfNoValueF{#1}{^{#1}}\qty(#2)}

\DeclareDocumentCommand{\Li}{oom}{\mathrm{Li}\IfNoValueF{#1}{_{#1}}\IfNoValueF{#2}{^{#2}}\qty(#3)}

\DeclareDocumentCommand{\denominator}{moo}{\qty(-#1^2 \IfNoValueF{#2}{+ #2^2}) \IfNoValueF{#3}{^{#3}}}
\DeclareDocumentCommand{\qmOne}{so}{\denominator{q_1}[m_1][\IfBooleanTF{#1}{\nu_1}{\IfNoValueF{#2}{#2}}]}
\DeclareDocumentCommand{\qmTwo}{so}{\denominator{q_2}[m_2][\IfBooleanTF{#1}{\nu_2}{\IfNoValueF{#2}{#2}}]}
\DeclareDocumentCommand{\qmThree}{so}{\denominator{q_{12}}[m_3][\IfBooleanTF{#1}{\nu_3}{\IfNoValueF{#2}{#2}}]}

\newcommand{\tildedp}[1]{\widetilde{\dd{#1}}}

% copied from https://tex.stackexchange.com/questions/2476/using-latex-to-render-hypergeometric-function-notation
\newmuskip\pFqmuskip
\newcommand*\pFq[6][8]{%
    \begingroup % only local assignments
        \pFqmuskip=#1mu\relax
        \mathchardef\normalcomma=\mathcode`,
        % make the comma math active
        \mathcode`\,=\string"8000
        % and define it to be \pFqcomma
        \begingroup\lccode`\~=`\,
        \lowercase{\endgroup\let~}\pFqcomma
        % typeset the formula
        {}_{#2}F_{#3}{\left(\genfrac..{0pt}{}{#4}{#5}\middle|#6\right)}%
    \endgroup
}
\newcommand{\pFqcomma}{{\normalcomma}\mskip\pFqmuskip}


